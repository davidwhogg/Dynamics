\documentclass[12pt]{article}
\usepackage{bm}

\newcounter{problem}
\newenvironment{problem}{%
  \subsubsection*{\refstepcounter{problem}Problem~\theproblem:}}{}
\renewcommand{\vector}[1]{\bm{#1}}
\newcommand{\tensor}[1]{\bm{\mathsf{#1}}}
\newcommand{\ihat}{\vector{i}}
\newcommand{\jhat}{\vector{j}}
\newcommand{\khat}{\vector{k}}

\begin{document}\sloppy\sloppypar\thispagestyle{empty}

\section*{PHYS-UA 120 Dynamics Problem Set 2}

\noindent
\emph{Due in the ``Dynamics'' hand-in box before noon on 2014 September 25.}

\begin{problem}
Kibble \& Berkshire, Ch 3, problem 1
\end{problem}

\begin{problem}
Kibble \& Berkshire, Ch 3, problem 14
\end{problem}

\begin{problem}
Kibble \& Berkshire, Ch 3, problem 15
\end{problem}

\begin{problem}
Imagine that the projectile launched in Kibble \& Berkshire, problem 6
was roughly a sphere of mass $M$ and radius $R$.  Roughly---and I mean
order of magnitude only---what conditions would you need to set on the
relationship between $M$ and $R$ such that the assumption (atmospheric
drag is negligible) would hold?  Your answer will have a ``$\gg$'' or
a ``$\ll$'' sign in it.  To get concrete, observe that atmospheric
drag is a ram-pressure force in a problem like this.

Once you have that answered, now imagine that the projectile is made
of cold iron.  Your result (above) can be made into a limit on the
mass, which is what?
\end{problem}

\end{document}

\documentclass[12pt]{article}
\usepackage{bm}

\newcounter{problem}
\newenvironment{problem}{%
  \subsubsection*{\refstepcounter{problem}Problem~\theproblem:}}{}
\renewcommand{\vector}[1]{\bm{#1}}
\newcommand{\tensor}[1]{\bm{\mathsf{#1}}}
\newcommand{\ihat}{\vector{i}}
\newcommand{\jhat}{\vector{j}}
\newcommand{\khat}{\vector{k}}

\begin{document}\sloppy\sloppypar\thispagestyle{empty}

\section*{PHYS-UA 120 Dynamics Problem Set 2}

\noindent
\emph{Due in the ``Dynamics'' hand-in box before noon on 2014 September 25.}

\begin{problem}
Kibble \& Berkshire, Ch 3, problem 1
\end{problem}

\begin{problem}
Kibble \& Berkshire, Ch 3, problem 14
\end{problem}

\begin{problem}
Kibble \& Berkshire, Ch 3, problem 15
\end{problem}

\begin{problem}
The equation (3.16) in Kibble \& Berkshire does \emph{not} apply to
the motion of a macroscopic object through the atmosphere.  Cross that
part out of your books!  What applies is actually ram-pressure force.
Also cross out the solution (3.17), which applies to no macroscopic
problem.

Imagine that the projectile launched in Kibble \& Berkshire, Ch~3,
problem~6 was roughly a sphere of mass $M$ and radius $R$.
Roughly---and I mean order of magnitude only---what conditions would
you need to set on the relationship between $M$ and $R$ such that the
assumption (atmospheric drag is negligible) would hold?  Your answer
will have a ``$\gg$'' or a ``$\ll$'' sign in it.  To get concrete, use
for the atmospheric drag the ram-pressure force, which depends only on
the density of the air (which you can estimate).

Once you have that answered, now imagine that the projectile is made
of solid iron.  Your result (above) can be made into a limit on the
mass, which is what?

Finally, what (formally) would have to be the size (radius) of the
iron projectile such that equation (3.16) would \emph{actually} hold?
That is, at what size would the viscous drag force be much stronger than
the ram-pressure force?  To find out, look up the viscosity of air and
Stokes Law.  Again, your answer will have a ``$\gg$'' or a ``$\ll$''
sign in it.  In that case, would the atmospheric drag be negligible?
\end{problem}

\end{document}

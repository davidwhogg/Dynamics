\documentclass[12pt]{article}
\usepackage{bm}

\newcounter{problem}
\newcounter{problempart}
\newenvironment{problem}{%
  \subsubsection*{\setcounter{problempart}{0}\refstepcounter{problem}Problem~\theproblem:}}{}
\newcommand{\problempart}{\par\refstepcounter{problempart}\textsl{(\alph{problempart})}~}
\renewcommand{\vector}[1]{\bm{#1}}
\newcommand{\tensor}[1]{\bm{\mathsf{#1}}}
\newcommand{\ihat}{\vector{i}}
\newcommand{\jhat}{\vector{j}}
\newcommand{\khat}{\vector{k}}

\begin{document}\sloppy\sloppypar\thispagestyle{empty}

\section*{PHYS-UA 120 Dynamics Problem Set 7}

\noindent
\emph{Due in the ``Dynamics'' hand-in box before noon on 2014 October 30.}

\begin{problem}
Kibble \& Berkshire, Ch 8, problem 2
\end{problem}

\begin{problem}
Kibble \& Berkshire, Ch 8, problem 9
\end{problem}

\begin{problem}
Kibble \& Berkshire, Ch 8, problem 10
\end{problem}

\begin{problem}
Kibble \& Berkshire, Ch 8, problem 15.
If you \emph{loved} this problem, just for \emph{fun} (not for credit),
consider doing problem~16 too.
Problem~16 is more important astrophysically!
\end{problem}

\end{document}

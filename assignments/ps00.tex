\documentclass[12pt]{article}
\usepackage{bm}

\newcounter{problem}
\newenvironment{problem}{%
  \subsubsection*{\refstepcounter{problem}Problem~\theproblem:}}{}
\renewcommand{\vector}[1]{\bm{#1}}
\newcommand{\tensor}[1]{\bm{\mathsf{#1}}}
\newcommand{\ihat}{\vector{i}}
\newcommand{\jhat}{\vector{j}}
\newcommand{\khat}{\vector{k}}

\begin{document}\sloppy\sloppypar\thispagestyle{empty}

\section*{PHYS-UA 120 Dynamics Problem Set 0}

\noindent
\emph{Due at the beginning of Lecture on 2014 September 09.}

\begin{problem}
Kibble \& Berkshire, Ch 1, problems 1 and 2
\end{problem}

\begin{problem}
Kibble \& Berkshire, Ch 1, problem 3
\end{problem}

\begin{problem}
Kibble \& Berkshire, Ch 1, problem 20
\end{problem}

\begin{problem}
In Chapter~1, page~5, Kibble \& Berkshire refer to the axis directions $\ihat, \jhat, \khat$ as ``unit vectors''.
In Appendix~A, Problem~15, they refer to a rotation matrix $\tensor{R}$ as a ``tensor''.
Give some reasons Prof Hogg might object to both of these characterizations.
\end{problem}

\end{document}

\documentclass[12pt]{article}
\usepackage{bm}

\newcounter{problem}
\newcounter{problempart}
\newenvironment{problem}{%
  \subsubsection*{\setcounter{problempart}{0}\refstepcounter{problem}Problem~\theproblem:}}{}
\newcommand{\problempart}{\par\refstepcounter{problempart}\textsl{(\alph{problempart})}~}
\renewcommand{\vector}[1]{\bm{#1}}
\newcommand{\tensor}[1]{\bm{\mathsf{#1}}}
\newcommand{\ihat}{\vector{i}}
\newcommand{\jhat}{\vector{j}}
\newcommand{\khat}{\vector{k}}

\begin{document}\sloppy\sloppypar\thispagestyle{empty}

\section*{PHYS-UA 120 Dynamics Problem Set 12}

\noindent
\emph{Due in the ``Dynamics'' hand-in box before noon on 2014 December 11.}

\begin{problem}
Kibble \& Berkshire, Ch 13, problem 4.
\end{problem}

\begin{problem}
Kibble \& Berkshire, Ch 13, problem 8.
\end{problem}

\begin{problem}
Write a computer program that more-or-less reproduces Fig.~13.12~(a), (b), and (c) in the book.
That is, integrate trajectories in the relevant equations, with one trajectory coming
to the limit cycle from the interior and one coming from the exterior in each case.
Write your own simple integrator, but
choose a step size that is small enough to make the integration accurate
at the level necessary for the figure.
Do your figures look like those in the book?  If there are discrepancies, what do you think is going on?
Hand in both your figures and the code that generated them.
\end{problem}

\end{document}

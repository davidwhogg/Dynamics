\documentclass[12pt]{article}
\usepackage{bm}

\newcounter{problem}
\newcounter{problempart}
\newenvironment{problem}{%
  \subsubsection*{\setcounter{problempart}{0}\refstepcounter{problem}Problem~\theproblem:}}{}
\newcommand{\problempart}{\par\refstepcounter{problempart}\textsl{(\alph{problempart})}~}
\renewcommand{\vector}[1]{\bm{#1}}
\newcommand{\tensor}[1]{\bm{\mathsf{#1}}}
\newcommand{\ihat}{\vector{i}}
\newcommand{\jhat}{\vector{j}}
\newcommand{\khat}{\vector{k}}

\begin{document}\sloppy\sloppypar\thispagestyle{empty}

\section*{PHYS-UA 120 Dynamics Problem Set 13}

\noindent
\emph{Due in the ``Dynamics'' hand-in box before noon on 2014 December 18.}

\begin{problem}
Integrate a trajectory starting at $t=0$ at
$x=0.100$, $y=0.800$, $v_x=-0.050$, $v_y=0.020$ for times $0<t<50$ in the
H\'enon--Heiles potential given on pp.~368--369 (with $m=1$).
Use a leap-frog integrator (as described in lecture).
Note that the textbook equations have bad units; the coordinates (including
time) have been implicitly made dimensionless.  Ignore that fact while
you solve the problem but then figure out (on your own; not for
credit) how that is possible.  Hand in a plot of
the trajectory in the $x$--$y$ plane plus a plot of the Hamiltonian vs
time (to show that your code is not obviously wrong).
Over-plot a second integration (using a different color or linestyle)
starting identically except with initial $x=0.101$.
Can you see the exponential divergence of trajectories?
Hand in your code along with the plots.
\end{problem}

\begin{problem}
Reproduce Fig.~D.2 as well as you can.  The lines are \emph{fixed}
points (or limit cycles) of the map, so at each value of $r$ you step
through, you have to iterate the map to ``convergence'' and then plot
some (say) 128 points.  When the limit is a one-cycle, these points
will all hit at the same place, when it is a two-cycle, in two places,
and so on.  Hand in your figure and your code.
\end{problem}

\end{document}

\documentclass[12pt]{article}
\usepackage{bm}

\newcounter{problem}
\newcounter{problempart}
\newenvironment{problem}{%
  \subsubsection*{\setcounter{problempart}{0}\refstepcounter{problem}Problem~\theproblem:}}{}
\newcommand{\problempart}{\par\refstepcounter{problempart}\textsl{(\alph{problempart})}~}
\renewcommand{\vector}[1]{\bm{#1}}
\newcommand{\tensor}[1]{\bm{\mathsf{#1}}}
\newcommand{\ihat}{\vector{i}}
\newcommand{\jhat}{\vector{j}}
\newcommand{\khat}{\vector{k}}

\begin{document}\sloppy\sloppypar\thispagestyle{empty}

\section*{PHYS-UA 120 Dynamics Problem Set 4}

\noindent
\emph{Due in the ``Dynamics'' hand-in box before noon on 2014 October 9.}

\begin{problem}
Kibble \& Berkshire, Ch 5, problem 2
\end{problem}

\begin{problem}
Kibble \& Berkshire, Ch 5, problem 7
\end{problem}

\begin{problem}
Kibble \& Berkshire, Ch 5, problem 20
\end{problem}

\begin{problem}
One consequence of the material in this Chapter is that the line or
trajectory described by a plumb line (a stationary string holding up a
weight) is not the same as any free-falling particle's trajectory.

\problempart Give the simplest possible explanation for the above
fact.
What is different about the falling, massive particle and the
stretched, hanging string?

\problempart Ignore air resistance in this part and all that follows.
Calculate the deviation between the trajectory of a particle of mass
$m$ dropped from rest at the top of the plumb line of length $\ell$
when the particle passes the bottom of the plumb line.
That is, by how much has the dropped mass veered away from the
``vertical'' line after length $\ell$?
Give your answer in terms of $m$, $\ell$, and the gravitational
acceleration $g$.

\problempart Now imagine that the mass was thrown \emph{downward}
(parallel to the plumb line) at speed $v$ instead of being released
from rest.
Will the deviation be greater or smaller than the deviation you
computed for something dropped from rest?
In principle, you don't have to compute anything to answer this; you
can just make a good argument.

\problempart Now ignore special relativity and imagine that you threw
the mass down at the \emph{speed of light}!
This speed is so high that $\ell\,g\ll c^2$ (so you can ignore
gravity!); now the deviation calculation is very easy!
Compute the deviation in this case.

\problempart \emph{(Bonus part, not for credit:)}
When we ``turn on'' special relativity, we would have to re-compute
the previous part, and for a laser pulse or laser beam (not a massive
particle).
When we do this, do you think it will increase or decrease the
computed deviation?  Why?
\end{problem}

\end{document}

\documentclass[12pt]{article}
\usepackage{bm}

\newcounter{problem}
\newenvironment{problem}{%
  \subsubsection*{\refstepcounter{problem}Problem~\theproblem:}}{}
\renewcommand{\vector}[1]{\bm{#1}}
\newcommand{\tensor}[1]{\bm{\mathsf{#1}}}
\newcommand{\ihat}{\vector{i}}
\newcommand{\jhat}{\vector{j}}
\newcommand{\khat}{\vector{k}}

\begin{document}\sloppy\sloppypar\thispagestyle{empty}

\section*{PHYS-UA 120 Dynamics Problem Set 1}

\noindent
\emph{Due in the ``Dynamics'' hand-in box before the beginning of Lecture on 2014 September 18.}

\begin{problem}
Kibble \& Berkshire, Ch 2, problem 1
\end{problem}

\begin{problem}
Kibble \& Berkshire, Ch 2, problem 16
\end{problem}

\begin{problem}
Kibble \& Berkshire, Ch 2, problem 29
\end{problem}

\begin{problem}
Prof Hogg did (or will) tap a mug in lecture.  What, approximately, is
the natural angular frequency $\omega_0$ and quality factor $Q$ for the
principal (most excited) mode of the mug?  Explain your reasoning.
Compare to the $Q$ of a piano string (hit a key and hold down the key
to hear the ``sustained'' note from the piano).
\end{problem}

\end{document}

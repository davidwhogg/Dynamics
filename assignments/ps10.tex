\documentclass[12pt]{article}
\usepackage{bm}

\newcounter{problem}
\newcounter{problempart}
\newenvironment{problem}{%
  \subsubsection*{\setcounter{problempart}{0}\refstepcounter{problem}Problem~\theproblem:}}{}
\newcommand{\problempart}{\par\refstepcounter{problempart}\textsl{(\alph{problempart})}~}
\renewcommand{\vector}[1]{\bm{#1}}
\newcommand{\tensor}[1]{\bm{\mathsf{#1}}}
\newcommand{\ihat}{\vector{i}}
\newcommand{\jhat}{\vector{j}}
\newcommand{\khat}{\vector{k}}

\begin{document}\sloppy\sloppypar\thispagestyle{empty}

\section*{PHYS-UA 120 Dynamics Problem Set 10}

\noindent
\emph{Due in the ``Dynamics'' hand-in box before noon on 2014 November 20.}

\begin{problem}
Complete the problem we started in class on Thursday November 13, with
two masses of mass $M$ connected to the walls with springs of spring
constant $k$ and to each other by a spring of constant $K$.
Write down the Lagrangian in each of the two generalized coordinate
systems we wrote down, the $\mu$ and $\kappa$ tensors, and then get
the eigenvalues and eigenvectors.
Do these confirm our expectations about $\omega^2$ and the
eigenvectors for both modes in both coordinate systems?
\end{problem}

\begin{problem}
Kibble \& Berkshire, Ch 11, problem 1
\end{problem}

\begin{problem}
Write a \texttt{Python} computer program that generates the matrix on
the left-hand side of equation~(11.33) in the book, for any arbitrary
number $n$.
Write also the code that obtains its eigenvalues and eigenvectors
(\texttt{numpy.linalg} is your friend here; it may be a one-liner).
Now plot the lowest-frequency (not highest) 4 modes (in the style of
Fig 11.6 in the book) for the case of $n=32$.
Label each plot with the computed frequency of that mode, and compare
it to the frequency you would expect for the corresponding mode from
the continuous string of length $[n+1]\,\ell$.
\end{problem}

\end{document}

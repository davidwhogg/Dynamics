\documentclass[12pt]{article}
\usepackage{bm}

\newcounter{problem}
\newenvironment{problem}{%
  \subsubsection*{\refstepcounter{problem}Problem~\theproblem:}}{}
\renewcommand{\vector}[1]{\bm{#1}}
\newcommand{\tensor}[1]{\bm{\mathsf{#1}}}
\newcommand{\ihat}{\vector{i}}
\newcommand{\jhat}{\vector{j}}
\newcommand{\khat}{\vector{k}}

\begin{document}\sloppy\sloppypar\thispagestyle{empty}

\section*{PHYS-UA 120 Dynamics Problem Set 3}

\noindent
\emph{Due in the ``Dynamics'' hand-in box before noon on 2014 October 2.}

\begin{problem}
Kibble \& Berkshire, Ch 4, problem 16
\end{problem}

\begin{problem}
Kibble \& Berkshire, Ch 4, problem 24
\end{problem}

\begin{problem}
Kibble \& Berkshire, Ch 4, problem 28
\end{problem}

\begin{problem}
Use the ideas in Kibble \& Berkshire, Ch~4, problem~18 to write a
Python program (using numpy and matplotlib, perhaps) that plots $x$ vs
$t$ and $y$ vs $t$ for the eccentric orbit described there.
Set $e = 0.55$ and $a = 1$\,AU and $\tau = 1$\,yr.
Make your time axis extend over $0<t<4$\,yr.
Label your axes!
You will have to solve Kepler's equation iteratively, I expect!
Hand in your code along with your plots.
\end{problem}

\end{document}

\documentclass[12pt]{article}
\usepackage{bm}

\newcounter{problem}
\newcounter{problempart}
\newenvironment{problem}{%
  \subsubsection*{\setcounter{problempart}{0}\refstepcounter{problem}Problem~\theproblem:}}{}
\newcommand{\problempart}{\par\refstepcounter{problempart}\textsl{(\alph{problempart})}~}
\renewcommand{\vector}[1]{\bm{#1}}
\newcommand{\tensor}[1]{\bm{\mathsf{#1}}}
\newcommand{\ihat}{\vector{i}}
\newcommand{\jhat}{\vector{j}}
\newcommand{\khat}{\vector{k}}

\begin{document}\sloppy\sloppypar\thispagestyle{empty}

\section*{PHYS-UA 120 Dynamics Problem Set 6}

\noindent
\emph{Due in the ``Dynamics'' hand-in box before noon on 2014 October 23.}

\begin{problem}
Kibble \& Berkshire, Ch 7, problem 3
\end{problem}

\begin{problem}
Kibble \& Berkshire, Ch 7, problem 4.
For this problem, assume that the system is in outer space,
not attached to anything and not subject to any gravitational forces.
\end{problem}

\end{document}
